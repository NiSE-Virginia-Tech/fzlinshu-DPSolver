\begin{abstract}
Constraint optimization problems (COP) on finite domains are typically solved via search. Many problems (e.g., 0-1 knapsack)
involve redundant search, 
making a general constraint solver revisit the same subproblems again and again. Existing approaches use caching, symmetry breaking, subproblem dominance, or search with decomposition to prune the search space of constraint problems. 
In this paper we present a different approach---\tool---which uses dynamic programming (DP) to efficiently solve certain types of constraint optimization problems (COPs). 
%to reduce the search space via dynamic programming. 
%cache and reuse subproblem solutions to opportunistically accelerate constraint solving. However, they do not examine whether the calculation of every subproblem is necessary. This paper introduces a novel approach---\tool---to further eliminate unnecessary subproblem resolution via dynamic programming. 
Given a COP modeled with MiniZinc, \tool first analyzes the model to decide whether the problem is efficiently solvable with DP. If so, \tool refactors the constraints and objective functions to model the problem as a DP problem.
%\tool introduces a new data structure (i.e., array), and refactor the constraints and
%refactor the constraints and objective functions. 
Finally, \tool feeds the refactored model to Gecode---a widely used constraint solver---for the optimal solution. 
Our evaluation shows that \tool significantly improves the performance of constraint solving. 

%yields a significant gain in search performance. 
%outperforms existing approaches (i.e., Gecode and Chuffed) with \todo{X-Y\%} speedups when searching for solutions. 

%when searching solutions for certain . 

%identifies subproblem dominance or caches subproblem solutions to accelerate constraint solving. 

%Many constraint models contain arrays or sets as variables. To deal with well-structured arrays, heuristic search algorithms used by general constraint solvers are feasible but not always efficient. Instead, dynamic programming often has the best performance on this task. Previous studies on dynamic programming generation focus on solving certain types of problems described by domain specific languages. Our paper proposes a novel approach for optimizing the general constraint solvers by using dynamic programming to assign values to arrays in the constraint model. To achieve this goal, we design techniques for automatically dividing the subproblems, checking optimal substructure, and dealing with overlapping estimation. We implement a fully automatic solver called \tool based on our approach. In our evaluation, we apply \tool to classical dynamic programming problems. The results show that our approach is much more efficient than general constraint solvers Gecode and \textsc{Chuffed}, and able to solve more types of problems compared with existing dynamic programming generation approaches.
\end{abstract}