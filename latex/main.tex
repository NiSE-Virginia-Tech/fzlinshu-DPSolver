%%%% ijcai19.tex

\typeout{IJCAI-19 Instructions for Authors}

% These are the instructions for authors for IJCAI-19.

\documentclass{article}
\pdfpagewidth=8.5in
\pdfpageheight=11in
% The file ijcai19.sty is NOT the same than previous years'
\usepackage{ijcai19}
\usepackage[utf8]{inputenc}
\usepackage[english]{babel}

% Use the postscript times font!
\usepackage{comment}
\usepackage{times}
\usepackage{soul}
\usepackage{url}
\usepackage[utf8]{inputenc}
\usepackage[small]{caption}
\usepackage{graphicx}
\usepackage{amsmath}
\usepackage{booktabs}
\usepackage{listings}
\usepackage{multirow}
\usepackage{multicol}
 \usepackage{mathrsfs}
\urlstyle{same}
\usepackage{xspace}
\usepackage{booktabs}
\usepackage{xcolor}
\usepackage{amsmath,amssymb,amsthm}
\usepackage[ruled,linesnumbered,boxed,vlined]{algorithm2e}
\lstset{
    basicstyle=\scriptsize\ttfamily,
    keywordstyle=\color{blue!70}\bfseries,
    commentstyle=\color{red!50!green!50!blue!50},
    frame=shadowbox,
    rulesepcolor=\color{red!20!green!20!blue!20},
    morekeywords={impose},
    escapeinside=``
}

\newcommand{\codefont}[1]{\footnotesize{\texttt{#1}}\normalsize}
\newcommand{\todo} [1]{\textcolor{blue}{{\sf TODO}: #1}}
%\newcommand{\tool}{\textsc{$ArraySolver$}\xspace}
\newcommand{\tool}{\textsc{$DPSolver$}\xspace}

\newtheorem{theorem}{Theorem}[section]
\newtheorem{corollary}{Corollary}[theorem]
\newtheorem{lemma}[theorem]{Lemma}
\renewcommand{\qedsymbol}{$\blacksquare$}

%\title{Opportunistic Constraint Optimization for Subproblem Reuse in Constraint Solving}
\title{
Optimizing Constraint Solving via Dynamic Programming
}

\author{
Shu Lin$^1$\footnote{Contact Author}\and
Na Meng$^2$\and
Wenxin Li$^1$\\
\affiliations
$^1$Department of Computer Science and Technology, Peking University, Beijing, China\\
$^2$Department of Computer Science, Virginia Tech, Blacksburg, VA, USA\\
\emails
\{fzlinshu,lwx\}@pku.edu.cn,
nm8247@vt.edu
}
%the following title seems to be too general
%\title{Exploiting Subproblem Reuse in Constraint Programming}

%\title{Optimizing Constraint Solving via Defining and Reusing Subproblems}

\begin{document}

\maketitle

\begin{abstract}
Constraint optimization problems (COP) on finite domains are typically solved via search. Many problems (e.g., 0-1 knapsack)
involve redundant search, 
making a general constraint solver revisit the same subproblems again and again. Existing approaches use caching, symmetry breaking, subproblem dominance, or search with decomposition to prune the search space of constraint problems. 
In this paper we present a different approach---\tool---which uses dynamic programming (DP) to efficiently solve certain types of constraint optimization problems (COPs). 
%to reduce the search space via dynamic programming. 
%cache and reuse subproblem solutions to opportunistically accelerate constraint solving. However, they do not examine whether the calculation of every subproblem is necessary. This paper introduces a novel approach---\tool---to further eliminate unnecessary subproblem resolution via dynamic programming. 
Given a COP modeled with MiniZinc, \tool first analyzes the model to decide whether the problem is efficiently solvable with DP. If so, \tool refactors the constraints and objective functions to model the problem as a DP problem.
%\tool introduces a new data structure (i.e., array), and refactor the constraints and
%refactor the constraints and objective functions. 
Finally, \tool feeds the refactored model to Gecode---a widely used constraint solver---for the optimal solution. 
Our evaluation shows that \tool significantly improves the performance of constraint solving. 

%yields a significant gain in search performance. 
%outperforms existing approaches (i.e., Gecode and Chuffed) with \todo{X-Y\%} speedups when searching for solutions. 

%when searching solutions for certain . 

%identifies subproblem dominance or caches subproblem solutions to accelerate constraint solving. 

%Many constraint models contain arrays or sets as variables. To deal with well-structured arrays, heuristic search algorithms used by general constraint solvers are feasible but not always efficient. Instead, dynamic programming often has the best performance on this task. Previous studies on dynamic programming generation focus on solving certain types of problems described by domain specific languages. Our paper proposes a novel approach for optimizing the general constraint solvers by using dynamic programming to assign values to arrays in the constraint model. To achieve this goal, we design techniques for automatically dividing the subproblems, checking optimal substructure, and dealing with overlapping estimation. We implement a fully automatic solver called \tool based on our approach. In our evaluation, we apply \tool to classical dynamic programming problems. The results show that our approach is much more efficient than general constraint solvers Gecode and \textsc{Chuffed}, and able to solve more types of problems compared with existing dynamic programming generation approaches.
\end{abstract}

\section{Introduction}
When solving a constraint optimization problem (COP), a general solver searches for an assignment of variables to (1) satisfy the constraints on those variables and (2) optimize an objective function. Such search may require repetitive computation when different branches on a search tree lead to the same subproblem. The unnecessary redundant work can cause the worst case complexity of search to be $O(m^n)$, where $n$ is the number of variables for assignments and $m$ is the number of branches on each node. 

Existing methods avoid or reduce redundant search via caching~\cite{Smith2005}, symmetry breaking~\cite{Gent2006}, subproblem dominance~\cite{chu2012exploiting}, problem  decomposition~\cite{kitching2007symmetric}, branch-and-bound pruning~\cite{marinescu2005and}, lazy clause generation~\cite{ohrimenko2009propagation}, or auto-tabling~\cite{dekker2017auto,zhou2015constraint}.

Dynamic programming (DP) is 
a classical method for solving complex problems. 
Given a problem, DP decomposes it into simpler subproblems, solves each subproblem once, stores their solutions with a table, and conducts table lookup when a subproblem reoccurs~\cite{Bertsekas:2000}. 
Although DP seems a nice search algorithm for COP solutions, we have not seen it to be used in solving general constraint models. 
Therefore, this paper explores how well DP helps optimize constraint solving. 
There are three major research challenges: 
%We were curious how well DP helps with constraint solving, so this paper focuses on optimizing constraint solving with DP. 

%To solve a COP via DP, we need to tackle three challenges:
%To optimize constraint solving via DP, we need to tackle three challenges: 

\begin{itemize}
    \item Given a COP, how can we automatically decide whether the problem is efficiently solvable with DP? 
    \item If a problem can be solved with DP, how can we implement the DP search?
    \item When there are multiple ways to conduct DP search for a problem, how can we automatically choose the best one with optimal search performance? 
\end{itemize}

To address these challenges, we designed and implemented a novel approach \tool, which opportunistically accelerates constraint solving in a non-intrusive way. 
Specifically, given a COP described in MiniZinc---a widely used constriant modeling language~\cite{nethercote2007minizinc}, \tool determines  whether the problem has (1) optimal substructures and (2) overlapping subproblems; if so, the problem is efficiently solvable with DP. Next, for each solvable problem, \tool
converts the original model to a DP-oriented model, such that a general constraint solver (i.e., Gecode~\cite{schulte2006gecode}) essentially conducts DP search when processing the new model. 
Third, if multiple DP-oriented models are feasible, \tool estimates the computation complexity of each model to choose the fastest one.  

We applied \tool to nine optimization problems, 
including seven DP problems and two non-DP ones. \tool significantly speeded up the constraint solving process for all problems.  
We also applied two state-of-the-art optimized constraint solvers---\textsc{ChuffedC} (caching) and \textsc{ChuffedL} (LCG)---to the same dataset for comparison, and observed that \tool significantly outperformed both techniques. We open sourced \tool and our evaluation data set at \url{https://github.com/fzlinshu/DPSolver}.

%\tool introduces negligible runtime overhead when it cannot solve a COP via DP


%	Combinatorial optimization problems widely exist in real world applications. Several constraint solving approaches \cite{Kuchcinski2013JaCoP,choco,schulte2006gecode} have been proposed to address combinatorial optimization problems. In these approaches, typically users first model the problems using dedicated constraint modeling languages, and then heuristic search algorithms are employed to solve these problems. 
	
%	However, a large class of combinatorial optimization problems are dynamic programming problems, or containing dynamic programming problems as subproblems. Typical dynamic programming problems include optimal consumption and saving problem in economics \cite{Weissensteiner2009A}, sequence alignment problem in genetics \cite{Needleman1970A}. While dynamic programming problems can be solved by heuristic search algorithm, the efficiency cannot match up to dedicated dynamic programming algorithms.
	
%	In this paper we aim to address problems or subproblems by dynamic programming algorithms in general constraint models. Our approach works by doing an additional analysis step before the solving process, and replaces some part of search by dynamic programming. The analysis step is static and the time is usually negligible compared to the problem solving time. And the replaced dynamic programming algorithm is much more efficient than the original search algorithm. Given a constraint model, our approach determines whether some arrays in this model are suitable to be dealt with by dynamic programming. If so, we generate dynamic programming algorithms to find values for these arrays, and report the time complexity and space complexity of the algorithms. The user could then employ policies to decide, based on the complexities of the generated dynamic programming algorithms, whether to use one of the algorithms, or to use a traditional solver to solve the whole problem.
	
\begin{comment}
    
	
    \subsection{Challenges}
	
    If a problem can be efficiently solved by dynamic programming, it must have optimal substructure and overlapping subproblems \cite{T2012Introduction}. Optimal substructure means that an optimal solution to a problem contains an optimal solution to subproblems. Overlapping subproblems means that a recursive algorithm to solve this problem will revisit the same subproblem over and over again. To determine whether a problem has the properties, we need to (1) first divide the problem into subproblems, and then show (2) the subproblems have optimal substructure and (3) are overlapping. However, all the three problems are challenging to solve.
	
	\begin{itemize}
	\item For the first problem, though there exist algorithms \cite{Bird1997Algebra,Morihata2014Dynamic} to divide a problem into subproblems, these algorithms require the problem to be represented as a fold-based functional program. It is still unknown how to divide a problem described by constraint modeling languages.
	
	\item For the second problem, though existing studies have defined sufficient necessary conditions for optimal substructure \cite{Moor1994Categories}, as far as we are aware, none of these conditions can be checked automatically. We still need to find conditions that can be checked automatically.
	
	\item The third problem is challenging because whether the subproblems overlap usually not only depends on the problem but also depends on the potential rules or patterns of the input data. With only problem description and the raw input data, it is often impossible to determine whether the subproblems overlap or not.

    \end{itemize}	
	
	\subsection{Our Solution}
	
	To address the three challenges, our approach includes the following novelty contributions. 
	\begin{itemize}
	    \item To address the first problem, we only concentrate on well-structured arrays in the constraint models. We further design an algorithm that divides the problem into subproblems along the index domain of the arrays.
	    \item To address the second problem, we only check whether an elementary function has the optimal substructure property by its derivative. Our algorithm checks a sufficient condition but nevertheless can be executed automatically. As our experiments will show, this sufficient condition is large enough to cover common dynamic programming problems.
	    \item Instead of trying to solve the third problem statically, which is impossible in most cases, our approach tries all combinations of arrays and produces a set of dynamic programming algorithms and their complexities. Then based on the input data of a problem, a solver could estimate the time and space consumption of each algorithm and then the selection is made by either the solver or the user.
	\end{itemize}

	\bigskip
	
	We implement a fully-automatic solver called \tool. \tool accepts a constraint model and data described in MiniZinc as input, detects dynamic programming subproblems, generates dynamic programming algorithms for subproblems, then solves the whole problem with search and one of the most suitable dynamic programming algorithm. The process can be done without any help from the user.
	
	In our experiment, we compare \tool with state-of-the-art constraint solvers, Gecode \cite{schulte2006gecode} and \textsc{Chuffed} \cite{chu2012exploiting} on a benchmark of several dynamic programming problems. The results show that our approach is much faster than the other solvers. We also performed a theoretical comparison between our approach and other related approaches, and identified three subclasses of dynamic programming problems that can only be solved by our approach but not other related approaches.
	
	\end{comment}
\section{A Motivating Example}
	To facilitate discussion, we introduce the 0-1 knapsack problem as an exemplar problem efficiently solvable with DP. 
	
	
	%In this section, we take 0-1 Knapsack problem as an example to present our approach. 0-1 Knapsack problem is described as follow:
	\begin{quote}
		\textbf{Problem Description:} There are $N$ items. There is a knapsack of capacity $C$. The $i^{th}$ item ($i \in [1,N]$) has value $V_i$ and weight $W_i$. Put a set of items $S \subseteq \{1,\ldots,N\}$ in the knapsack, such that the sum of the weights is at most $C$ while the sum of values is maximal.  
		%Given a knapsack with  limited capacity C, and N items each has a positive profit and a positive weight. The goal is, select a set of items to put into the knapsack so that the total weight is less than or equal to the capacity of the knapsack, and the total profit of these items is as large as possible.
	\end{quote}
\noindent	
The 0-1 knapsack problem is a typical COP and can be modeled with MiniZinc in the following way (see Figure~\ref{fig:knapsack}):
 
	\begin{figure}[htb]
\begin{lstlisting}[frame=single]
% Input arguments
    int: N;
    int: C;
    array[1..N] of int: V;
    array[1..N] of int: W;

% Variables
    var set of 1..N: knapsack;

% Constraints
    constraint sum (i in knapsack)(W[i]) <= C;

% Objective function
    solve maximize sum (i in knapsack)(V[i]);
\end{lstlisting}
\caption{Model of the 0-1 knapsack problem}\label{fig:knapsack}
\end{figure}
\noindent
Given the above model description, a general  constraint solver (e.g., Gecode) typically enumerates all possible subsets of $N$ to find the maximum value summation. Such na{\"i}ve search has $O(2^N)$ time complexity. Our research intends to significantly reduce this complexity. %to $O(NC)$.       
    
 
 \section{Approach}\label{sec:approach}
 
 \tool consists of three phases: DP problem recognition (Section~\ref{sec:recognize}), DP-oriented model description generation (Section~\ref{sec:generate}), and description selection (Section~\ref{sec:select}). 
 Phase II is taken only when Phase I identifies one or more solvable problems in a given MiniZinc model; and Phase III is taken only when Phase II generates multiple alternative models. 
 
 \subsection{Phase I: DP Problem Recognition}
 \label{sec:recognize}
If a problem can be efficiently solved with dynamic programming, it must have two  properties~\cite{T2012Introduction}:

\begin{enumerate}
    \item[P1.] \textbf{Optimal Substructures.} An optimal solution can be constructed from optimal solutions of its subproblems. This property ensures the usability of DP, because DP saves and uses only the optimal instead of all solutions to subproblems for optima calculation. 
    
   \item[P2.] \textbf{Overlapping Subproblems.} 
   When a problem is decomposed into subproblems, some  subproblems are repetitively solved. 
   This property ensures the usefulness of DP, because by memoizing solutions to subproblems, DP can eliminate repetitive computation. 
\end{enumerate}
Essentially, DP is applicable to a COP when optimal solutions to subproblems can be repetitively reused for optima calculation.
%non-optimal solutions to subproblems cannot lead to any optimal solution to the overall problem, and (2) different branches in a solution search tree overlap. DP optimizes constraint solving by pruning unpromising and redundant search paths. 
If a COP has both properties, we name it a 
\textbf{\emph{DP problem}}.  
 
 Given a COP modeled in MiniZinc, \tool recognizes a DP problem by taking three steps: 1) identifying 
 any array variable and accumulative function applied to those array elements, 
 2) converting constraints and objective functions to recursive functions of array elements (i.e., subproblems), and 
 3) checking recursive functions for the two properties. 

\paragraph*{Step 1: Candidate Identification}
\tool checks for any declared variable of the array data type because DP is usually applied to arrays. Additionally, if a variable is a set, as shown by the \codefont{knapsack} variable in Figure~\ref{fig:knapsack}, \tool converts it to a boolean array \codefont{b} such that ``\codefont{b[i]}'' indicates whether the $i^{th}$ item in the set is chosen or not. 
By doing so, \tool can also handle problems with set variables. We name the identified array or set variables \textbf{\emph{candidate variables}}. 
 
Next, \tool checks 
whether any candidate variable is used in at least one constraint and one objective function; if so, \tool may find optimization opportunities when enumerating all value assignments to the variable. 
In Figure~\ref{fig:knapsack}, both the constraint and objective can be treated as functions of the newly created array $b$ as below: 
\begin{equation}
\label{eq1}
   \sum_{i=1}^{N}b[i]*W[i] \le C \text{, where }b[i]\in\{0, 1\}  \tag{3.1}
\end{equation}
\begin{equation}
\label{eq2}
    maximize \sum_{i=1}^{N}b[i]*V[i] \text{, where }b[i]\in\{0, 1\}  \tag{3.2}
\end{equation}
When these functions have any accumulative operator (e.g., $sum$), it is feasible to further break down the problem into subproblems.
Thus, \tool treats the variable \codefont{b} together with related functions as a candidate for DP application.

\paragraph{Step 2: Function Conversion}
\tool tentatively converts relevant constraint and objective functions to step-wise recursive functions in order to identify subproblems and prepare for further property check. 
%Actually, this conversion procedure has two parts: from accumulative functions to recursive functions, and from recursive functions to DP-oriented recursive functions. 
%$(a)$ {$Accumulative\longrightarrow Recursive$.} 
Specifically, 
\tool unfolds all accumulative functions 
%to recursive functions and breaks down the given problem into simpler subproblems 
recursively. 
For instance, the constraint formula (\ref{eq1}) can be converted to
\begin{align}
\label{fun1}
f_0(W, b)&=0 \nonumber\\
f_1(W, b)&=f_0(W, b) + b[1] * W[1] \nonumber\\
c_1(W, b)&=C-f_1(W, b)\nonumber\\
c_1(W, b)&\ge 0 \nonumber\\
\ldots \tag{3.3}\\  
f_N(W, b)&=f_{N-1}(W, b) + b[N] * W[N] \nonumber\\
c_N(W, b)&=C-f_N(W, b) \nonumber\\
c_N(W, b)& \ge 0 \nonumber
\end{align}

Here, $f_i(W, b) (i\in[1, N])$ computes weight sums for the first $i$ items given (1) the item weight array $W$ and (2) the boolean array $b$.
The function $c_i(W, b)$ subtracts $f_i(W, b)$ from capacity $C$, to define the constraint for each subproblem. Here, $c_i$ means ``\emph{the remaining capacity limit for the last $(N-i)$ items}''. 
The value of $c_i$ helps decide whether an item should be added to the knapsack. Namely, if the weight of the $(i+1)^{th}$ item is greater than $c_i$'s value, the item should not be added.
%The value of $c_i$ is limited by the lower bound $(N - i) * min\{W[1], \ldots, W[N]\}$.
%, where $w$ is the minimum weight among all items. \tool scanned the input data to obtain $w > 0$, and simplified the lower bound to $0$. 
%The $c_i$ function essentially defines the constraint for each subproblem. 
%Here, $c_i$ means ``the remaining capacity limit for the last $(N-i)$ item''. 
%The value of $c_i$ helps decide whether an item can be added to the knapsack. Namely, if the weight of $(i+1)^{th}$ item is greater than $c_i$, this item cannot be added.
Actually, the $c_i$'s value is limited by the lower bound $min\{0, (N - i) \cdot min\{W[1], \ldots, W[N]\}\}$. Because \tool scanned the input data and found all weight values to be greater than 0, the lower bound used in (\ref{fun1}) was simplified to 0.

Similarly, the objective function (\ref{eq2}) can be transferred to
\begin{align}
\label{fun2}
o_0(V, b)&=0 \nonumber\\
o_1(V, b)&=o_0(V, b) + b[1] * V[1]\nonumber\\
opt_1(V)&=max \text{ }o_1(V, b) \nonumber\\
\ldots \tag{3.4}\\
o_N(V, b)&=o_{N-1}(V, b) + b[N] * V[N]\nonumber\\
opt_N(V)&=max\text{ }o_N(V, b) \nonumber
%optimal[i][j]&=\left\{
%\right.
\end{align}

Here, $o_N(V, b)$ calculates the value summation.
The function $opt_i(V)$ defines the optimization goal for each subproblem related to the first $i$ items. When all possible value assignments are explored for $b$, $o_N(V, b)$ and $opt_N(V)$ functions can be executed to obtain the maximal summation. 

%$o1(V, b)$ can be used to produce the value summations this function produces the value summations accordingly to obtain the maximal summation. 

%records the value summations of some items. As with the original accumulative functions (Formula~\ref{eq1} and~\ref{eq2}),  these recursive functions still model the exhaustive search for subsets of N items. 


\paragraph{Step 3: Property Check}
With the converted functions, \tool checks for two properties in sequence. 

\paragraph{$(a)$ \emph{Verifying optimal substructures.}}
We first defined and proved the following theorem:

\begin{theorem}
\label{thm1}
Given two sets of functions, $O=\{o_0(\cdot), o_1(\cdot), o_2(\cdot),  \ldots, o_n(\cdot)\}$ and  $Opt=\{opt_1(\cdot), opt_2(\cdot), \ldots, opt_n(\cdot)\}$ (``$\cdot$'' is a placeholder for arguments), for any $i\in[1, n]$, suppose that
\begin{itemize}
    \item $o_i(\cdot)=h(o_{i-1}(\cdot), b[i])$ where $b$ is a candidate variable and h is monotonically increasing in $o_{i-1}(\cdot)$,  
    \item $opt_i(\cdot)=max\text{ }o_i(\cdot)$. 
    %, and \item $l(h(\cdot))=h(l(\cdot))$, namely, $l$ and $h$ are commutative.
\end{itemize}
Then $opt_i(\cdot)$ is monotonically increasing in $opt_{i-1}(\cdot)$. 
\end{theorem}

In this theorem, each $o_i(.)$ is a function, representing all possible value sums produced when $b[1..i]$ is assigned with different vector values, while $max\text{ }o_i(.)$ is a value, representing the maximum among those value sums.

\begin{proof}
For any $i\in[1, n]$, 
%since $o_{i-1}(\cdot) \le max \text{ }o_{i-1}(\cdot)$ and $h$ increases monotonically, we deduce that $h(o_{i-1}(\cdot))\le h(max \text{ }o_{i-1}(\cdot))$, namely, $o_i\le h(max \text{ }o_{i-1}(\cdot))$. As $h(max\text{ }o_{i-1}(\cdot))$ is the maximum value that $o_i(\cdot)$ can obtain, we have $max\text{ }o_i(\cdot)=h(max \text{ }o_{i-1}(\cdot))$.  
\begin{align}
    &\because o_{i-1}(\cdot) \le max \text{ }o_{i-1}(\cdot) \nonumber\\
    & \therefore h(o_{i-1}(\cdot), b[i])\le h(max \text{ }o_{i-1}(\cdot), b[i]) \nonumber\\
     &\therefore o_i(\cdot) \le h(max \text{ }o_{i-1}(\cdot), b[i])\nonumber\\ %(o_i(\cdot)=h(o_{i-1}(\cdot)))\nonumber\\
     &\therefore max\text{ }o_i(\cdot)=h(max \text{ }o_{i-1}(\cdot), b[i]), i.e.,  \nonumber \\
     & opt_i(\cdot)=h(opt_{i-1}(\cdot), b[i]) \nonumber
     %& \therefore l(max\text{ }o_i(\cdot))=l(h(max \text{ }o_{i-1}(\cdot)))\nonumber\\
     %&\therefore opt_i(\cdot)=l(h(max \text{ }o_{i-1}(\cdot)))\nonumber\\
     %&\therefore opt_i(\cdot)=h(l(max\text{ }o_{i-1}(\cdot))) (h \text{ and }l\text{ are commutative})\nonumber\\
    %&\therefore opt_i(\cdot)=h(opt_{i-1}(\cdot))\nonumber
\end{align}
Therefore, $opt_i(\cdot)$ monotonically increases in $opt_{i-1}(\cdot)$. The optimal solution can be composed with the optimal solution to a subproblem. \qedhere  
\end{proof}

Similarly, we defined and proved a related theorem when the $max$ function used in Theorem~\ref{thm1} is replaced with $min$. Furthermore, there are problems whose $opt_i(\cdot)$ functions are expressions of $max\text{ }o_i$ (e.g., $max\text{ }o_i+3$) instead of $max\text{ }o_i$ itself. 
To ensure the generalizability of our approach, we also consider such problems 
to have optimal substructures as long as $max\text{ }o_i$ is a function of $max\text{ }o_{i-1}$. 
This is because when the \textbf{\emph{extreme value}} ($max$ or $min$) related to a problem's optimal solution can be computed 
with the extreme values derived for subproblems,
we can always construct the optimal solution by reusing extreme values from subproblems. 

%we still generally consider such problems to have optimal substructures, because the optimal solution can be computed based on some maximum values computed for subproblems.  

Based on the above theorems, given converted objective functions, \tool locates the used $max$ or $min$ function and tentatively matches $h$. 
For each matched $h$ function, \tool takes the derivative to check for any monotonicity property.
%For each matched $l$, \tool checks whether the evaluation order between $l$ and $h$ can be swapped. 
For our example (function sets (\ref{fun2})), we have 
\begin{align}
    & h_{0/1}(o_{i-1}(\cdot))=o_{i-1}(\cdot)+b[i]*V[i]. \nonumber
   % & l_{0/1}(x)=x
\end{align}
Assuming $h_{0/1}$ to be continuous, \tool finds the derivative to be $\partial o_i(\cdot)/\partial o_{i-1}(\cdot)=1>0$. Therefore, $h_{0/1}$ increases monotonically in $o_{i-1}$. Solutions of the 0-1 knapsack problem have   optimal substructures.  
%Besides, \tool randomly samples multiple values of $b$ and compares the results of $l_{0/1}(h_{0/1}(\cdot)$ and $h_{0/1}(l_{0/1}(\cdot))$. Since these composed functions always produce the same values, \tool concludes that the two functions are commutative.  

%If a problem does not pass this property check, \tool stops its attempt to optimize constraint solving; otherwise, it continues with a second property check. 

\paragraph{$(b)$ \emph{Verifying overlapping subproblems.}} We defined and proved another theorem to facilitate property checking. 

\begin{theorem}
\label{thm2}
Given two sets of functions, $F=\{f_0(\cdot), f_1(\cdot), \ldots, f_n(\cdot)\}$ and $Con=\{c_0(\cdot), c_1(\cdot), \ldots, c_n(\cdot)\}$ (``$\cdot$'' is a placeholder for arguments), for 
any $i\in[1, n]$, suppose that 
\begin{itemize}
\item $f_0(\cdot)=v_0$ where $v_0$ is a constant, 
\item $f_i(\cdot)=p(f_{i-1}(\cdot), b[i])$ where $b$ is a variable, and
\item $c_i(\cdot)=q(f_{i-1}(\cdot), b[i])$. 
\end{itemize}
Then there exist overlapping subproblems of $f_n(\cdot)$ and $c_n(\cdot)$ 
between different value assignments of $b$.
\end{theorem}

\begin{proof}
This theorem includes two parts: 
(1) $f_n(\cdot)$ has overlapping subproblems; and (2)  $c_n(\cdot)$ has overlapping subproblems. Here we demonstrate the proof by induction for $f_n(\cdot)$. 
The proof for $c_n(\cdot)$ is similar. 

\begin{enumerate}
    \item \textbf{n=2:} For any two  assignments of $b$: $b_{A2}'$ and $b_{A2}''$, where $b_{A2}'\neq b_{A2}''$ but $b_{A2}'[1]=b_{A2}''[1]$, 
\begin{align}
    & \because \text{The first elements of both arrays are identical,}\nonumber\\
    & \therefore \text{The evaluation procedures of }
    f_1(\cdot)=p(f_0(\cdot), b[1]) \nonumber\\
    &\text{remains the same given $b_{A2}'$ and $b_{A2}''$. } \nonumber\\
    & \therefore \text{When $f_2(\cdot)$ is computed based on $f_1(\cdot)$, the } \nonumber\\
    & \text{calculation procedure of $f_1(\cdot)$ overlaps between $b_{A2}'$ } \nonumber\\
    &\text{and $b_{A2}''$. Thus, $f_n(\cdot)$ has overlapping subproblems.} \nonumber
\end{align}
    \item \textbf{n=k (k$>$2):} Assume the theorem to hold. Namely, there exist two assignments $b_{Ak}'$ and $b_{Ak}''$ ($b_{Ak}'\neq b_{Ak}''$), between which the evaluation procedures of  $f_n(\cdot)$ have overlapping subproblems. 
    \item \textbf{n=k+1:} To prove the theorem, we compose two assignments: $b_{A(k+1)}'$ and $b_{A(k+1)}''$, such that $b_{A(k+1)}'[1:k]=b_{Ak}'$ and $b_{A(k+1)}''[1:k]=b_{Ak}''$. %$b_{A(k+1)}'[k+1]=b_{A(k+1)}''[k+1]$. 
    Intuitively, these arrays separately include $b_{Ak}'$ and $b_{Ak}''$ as the first $k$ elements.  
\begin{align}
    & \because f_{k+1}(\cdot)=p(f_{k}(\cdot), b[k+1]) \nonumber \\
    %& \because %c_{k+1}(\cdot)=q(f_{k}(\cdot), b[k+1])\nonumber\\
   & \therefore \text{The function depends on the evaluation result of $f_{k}(\cdot)$,}\nonumber\\
   & \text{a function computed based on the first $k$ elements} \nonumber \\
   & \because \text{ The evaluation processes of $f_k(\cdot)$ between $b_{Ak}'$ and $b_{Ak}''$} \nonumber \\
   & \text{have overlapping subproblems} \nonumber \\
   & \therefore \text{The evaluation processes of $f_{k+1}(\cdot)$ between $b_{A(k+1)'}$}\nonumber \\
   &\text{ and $b_{A(k+1)}''$ also overlap.} \nonumber 
\end{align}    
\end{enumerate}
Therefore, when different value assignments of $b$ are explored, there are overlapping subproblems to resolve. 
\end{proof}

Based on Theorem~\ref{thm2}, given converted constraint functions, \tool tries to match $p$ and $q$ by using or unfolding the formulas of 
 $f$ and $c$ functions. Thus, for our example (function sets (\ref{fun1})), the matched functions are
\begin{align}
    & p_{0/1}(f_{i-1}(\cdot), b[i])=f_{i-1}(\cdot)+b[i]*W[i], \nonumber\\
    & q_{0/1}(f_{i-1}(\cdot), b[i])=C-f_{i-1}(\cdot)-b[i]*W[i]. \nonumber
   % & l_{0/1}(x)=x
\end{align}
Notice that $p_{0/1}$ is derived from the $f$ formulas, while $q_{0/1}$ is obtained when \tool replaces the occurrence of $f$ in $c$ formulas. 
 With such matched functions found, the 0-1 knapsack problem passes the property check. 
 
DP trades space for time by storing and reusing optimal solutions to subproblems. 
If a problem has the first property only, \tool does not proceed to Phase II. Because there is no reuse of intermediate results, the extra space consumption for caching those results will not bring any execution speedup. 
However, if a problem has the second property only, even though DP is not applicable, \tool still creates a table to memoize \emph{all} intermediate results  for data reuse and runtime overhead reduction. 

  
 
%When a problem passes both property checks, \tool determines the problem to be a DP problem. If a problem only has the first property, even though DP is applicable, we do not apply it because 
 
 \subsection{Phase II: DP-Oriented Model Generation}
 \label{sec:generate}
 
%To efficiently solve a DP problem with a general-purpose constraint solver, \tool rewrites  
The above-mentioned recursive functions (e.g., function sets (\ref{fun1}) and (\ref{fun2})) are not directly usable by DP for efficient search because no redundant computation is eliminated. 
%of redundant computation. For instance, given two value assignments of $b$ (i.e., $b_N'$ and $b_N''$) that have identical values for the first $(N-1)$ items,
%$f_{N-1}(W, b)$ repeats the same computation procedure and outputs the same value. To eliminate redundant computation, 
We need to rewrite those functions 
such that intermediate results for subproblems are stored to a table and are used to replace repetitive computation. 
%The reason is that a DP algorithm usually leverages a table to record solutions for subproblems and reuse these solutions to handle bigger subproblems. 

Specifically, given (1) a candidate variable $b$, (2) a series of constraint functions $c(\cdot)$, (3) a series of objective functions $o(\cdot)$, and (4) the extreme value we care about (i.e., $max$ or $min$), \tool creates a two-dimension array $M$ such that 
\begin{align}
    M[i, c_i]&=extreme\text{ }o_i(\cdot) \nonumber\\
           &=extreme\text{ }h(extreme\text{ }o_{i-1}(\cdot), b[i]) \nonumber \\
           &=extreme\text{ }h(M[i-1, c_{i-1}], b[i]) \tag{3.5}
\end{align}
In $M$, $i$ corresponds to the array index range of $b$,
$c_i$ represents the corresponding valid constraint value, while the cell $M[i, c_i]$ saves the extreme value computed for each subproblem, meaning ``\emph{given the first i elements and the constraint value $c_i$, what is the extreme value of $o_i$}?''. When different values of $b[i]$ are enumerated, the corresponding $c_{i-1}$ can be different. Thus, multiple cells in the row $M[i-1]$ may be reused for computation, and $M[i, c_i]$ is usually decided
by the value comparison between these cells. 
By generating such table-driven recursive functions to model a DP problem, \tool can produce a DP-oriented MiniZinc description. 

Figure~\ref{fig:knapsack2} shows the newly generated model for our example. 
The space complexity of creating the $dpvalue$ table is $O(NC)$, while the time complexity is $O(NC)$. This time complexity refers to the whole resolution process because the DP-oriented model is resolved by propagation only.
%\todo{1. Is this MiniZinc model? 2. More questions concerning the nested 0-1 knapsack problem (space/time complexity). 3. Without optimal substructure, is there any optimization? 4. Problems with after effect. How are the results saved?} 

	\begin{figure}[htb]
\begin{lstlisting}[frame=single]
% Input arguments
    ... % unchanged N, C, V, W

% Variables
    array[0..N, 0..C] of var int: dpvalue;

% Constraints
    constraint forall(j in 0..C)
      (dpvalue[0,j] = if j==0 then 0 else -1 endif);
    function var int: calcValue(int: i, int: j, int: k) =
      if j-W[i]*k>=0 then
        if dpvalue[i-1,j-W[i]*k] != -1 then
          dpvalue[i-1,j-W[i]*k]+V[i]*k
        else -1 endif
      else -1 endif;
    constraint
      forall(i in 1..N, j in 0..C) (
        dpvalue[i,j] = max(k in 0..1)(calcValue(i,j,k))
      );

% Objective function    
    solve maximize max(j in 0..C)(dpvalue[N,j]);
\end{lstlisting}
\caption{DP-oriented model of the 0-1 knapsack problem}\label{fig:knapsack2}
\end{figure}

\textbf{Handling of Side Constraints.} 
Some COP models are defined with \emph{side constraints} in addition to accumulative functions. \tool can still process such models by converting each side constraint to an \codefont{if}-clause. 
For instance, a variant of the 0-1 knapsack problem can have an additional requirement as ``\emph{Item 3 must be put in the knapsack}'', which can be expressed as a side constraint ``\codefont{3 in knapsack}''. For this side constraint, \tool generates an \codefont{if}-clause ``\codefont{if i==3 $/\backslash$ k!=1 then -1 else ...}'' and inserts it to the beginning of the \codefont{calcValue(...)} function in Figure \ref{fig:knapsack2}.

%In addition, for other constraints without accumulative functions, \tool simply converts each constraint to an ``if-clause'', and then inserts it to the ``calcValue(...)'' function. For example, if we add a side constraint ``\texttt{3 in knapsack}'' to the 0-1 knapsack model, it will be converted to ``\texttt{if i==3 then if k!=1 then 0 else ...}'' and inserted to the beginning of the ``calcValue(...)'' function in Figure \ref{fig:knapsack2}.
 
 
 \subsection{Phase III: DP-Oriented Model Selection}
 \label{sec:select}
 
 For some COP problems, \tool can create multiple alternative optimization strategies. Given the space limit (e.g., 4GB) specified by users when they run the tool via commands, \tool automatically estimates the time/space complexity of each alternative, chooses the models whose table sizes are within the space limit, and then recommends the model with the maximum speedup.  
 
To better explain this phase, we take the nested 0-1 knapsack problem as another exemplar DP problem (see Figure~\ref{fig:knapsack3}).  
 	\begin{figure}[htb]
\begin{lstlisting}[frame=single]
% Input arguments
    int: N1, N2, C1;
    array[1..N1] of int: V1, W1;
    array[1..N2] of int: V2, W2;

% Variables
    var int: C2;
    var set of 1..N1: K1;
    var set of 1..N2: K2;

% Constraints
    constraint sum (i in K1)(W1[i]) <= C1;
    constraint C2 = sum (i in K1)(V1[i]);
    constraint sum (i in K2)(W2[i]) <= C2;
    
% Objective function
    solve maximize sum (i in K2)(V2[i]);
\end{lstlisting}
\caption{Model of the nested 0-1 knapsack problem. There are two sets of items ($N1$ and $N2$).
Choose items in both sets to separately fill two knapsacks: $K1$ and $K2$. If the value sum in $K1$ is used as the capacity of $K2$, try to maximize the value sum in $K2$.}\label{fig:knapsack3}
\end{figure}

When analyzing the problem, \tool identifies two candidate variables---$b_1$ and $b_2$---to separately represent subsets $K1$ and $K2$.
%---the boolean array reflecting different subsets of $N1$, and $b2$---another boolean array corresponding to subsets of $N2$. 
Since both variables are related to constraints, \tool explores and assesses three potential ways to create DP-oriented models: 

 \begin{enumerate}
     \item[(a)] Optimization based on $b_2$: %We use $C2$ to record the capacity of $K2$ (i.e., the maximum value sum in $K1$). 
     When exhaustive search is leveraged to
     fill $K1$ in various ways, for each produced $C2$ value, the optimization problem becomes a regular 0-1 knapsack problem---a DP problem. Therefore, \tool can create a DP-oriented model, with both time and space complexity as $O(N2C2)$. Since the exhaustive search handles such 0-1 knapsack problems with $O(2^{N2})$ time complexity, the optimization speedup is estimated as $(2^{N2}/N2C2)$.
     
      \item[(b)] Optimization based on $b_1$: When exhaustive search is used to fill $K2$ in various ways, for each generated weight sum $sum_w$ in $K2$, we need the value sum in $K1$ (i.e., C2) to be no less than $sum_w$ while the weight sum in $K1$ to be no more than $C1$. Because these converted problems do not have any objective function, they are not considered as DP problems. \tool does not create any model for optimization. 
     
     \item[(c)] Optimization based on $b_1$ and $b_2$: \tool concatenates $b_1$ and $b_2$ to create a larger array $b_3$. In $b_3$, the first $N1$ elements are involved in the first two constraints, while the last $N2$ elements are related to the third constraint and objective function. Thus, $b_3$ is still a candidate variable and the given problem is a DP problem. The exhaustive search obtains $O(2^{N1}2^{N2})$ time complexity. In comparison, DP search has both time and space complexity as 
     $O((N1+N2)C1Sum^2)$ time complexity, where $Sum$ is the upper bound of $sum_w$. Consequently, the estimated speedup is $(2^{N1}2^{N2}/\left[(N1+N2)C1Sum^2\right])$. 
 \end{enumerate}
 
 	\begin{figure}[htb]
\begin{lstlisting}[frame=single]
% Input arguments
    ... % unchanged N1, N2, C1, V1, W1, V2, W2
    int: sum = 20;
    % the upper bound of sum_w obtained from the input

% Variables
    array[0..N1+N2,0..C1,0..sum,0..sum] of var int:
      dpvalue;

% Constraints
    constraint
      forall(i2 in 0..C1, i3 in 0..sum, i4 in 0..sum) (
          dpvalue[0,i2,i3,i4] =
            if i2==0/\i3==0/\i4==0 then 0 else -1 endif
      );
    function var int: calcValue(int: i1, int: i2, int: i3,
                                int: i4, int: i5) =
      if i1 <= N1 then
        if i2-W1[i1]*i5>=0 /\ i3-V1[i1]*i5>=0 then
          dpvalue[i1-1,i2-W1[i1]*i5,i3-V1[i1]*i5,i4]
        else -1 endif
      elseif i4<=i3 /\ i4-W1[i1-N1]*i5>=0 then
        if dpvalue[i1-1,i2,i3,i4-W1[i1-N1]*i5] != -1 then
          dpvalue[i1-1,i2,i3,i4-W1[i1-N1]*i5]+V2[i1-N1]*i5
        else -1 endif
      else -1 endif;
    constraint
      forall(i1 in 1..N1+N2, i2 in 0..C1,
             i3 in 0..sum, i4 in 0..sum) (
        dpvalue[i1,i2,i3,i4] =
          max(i5 in 0..1)(calcValue(i1,i2,i3,i4,i5))
      );

% Objective function    
    solve maximize
      max(i2 in 0..C1, i3 in 0..sum, i4 in 0..sum)
         (dpvalue[N1+N2,i2,i3,i4]);
\end{lstlisting}
\caption{DP-oriented model of the nested 0-1 knapsack problem when
the optimization is based on both $b1$ and $b2$}\label{fig:knapsack4}
\end{figure}

 When the two separately generated models described in (a) and (c) both have their tables smaller than the space limit, \tool suggests the one with more speedup (i.e., (c)).
  Figure \ref{fig:knapsack4} shows the DP-oriented model resulted from (c).
  Intuitively, the more array variables are involved in optimization (e.g., $(b1, b2)$ vs.~$b2$), the more performance gain \tool is likely to obtain.  
\section{Implementation}
\label{sec:impl}

\tool is implemented based on a widely used  open-source constraint solver Gecode~\cite{schulte2006gecode}. 
Given a user-provided MiniZinc model, \tool analyzes the problem, converts the model to a DP-oriented model when possible, and passes the model to Gecode. 
If \tool cannot optimize a model, it passes the original model to Gecode. 
If multiple alternative DP-oriented models are generated, \tool passes the best one. 

For some problems, \tool implements extra handling to optimize constraint solving as much as possible. 

	\begin{description}
		\item[Problems with aftereffects.] 
		Some problems (e.g., longest increasing subsequence) have the \textbf{\emph{aftereffect property}}. Namely, the value of $o_i(\cdot)$ not only depends on $o_{i-i}(\cdot)$ or $b[i]$, but also on elements in $b[1:i-1]$ (e.g., $b[i-1]$). \tool handles such problems by saving more data for each subproblem. If the computation of $o_i(\cdot)$ depends on last $T$ elements in $b$, i.e., $b[i-T:i-1]$, then \tool increases the table by $T$ dimensions to save the involved $T$ elements.  
		%	has aftereffect property because the value of the next subproblem depends upon not only the current subproblem but also the previous one. When a problem has aftereffect, the original thinning method cannot be used for transforming constraints into recurrence equations. One possible solution is, rewriting variables and constraints in another form to avoid aftereffect, which is hard to be accomplished automatically.
		
		%In our approach, we introduce a Side Effect Constant $T$ during transforming. By storing the values of the last $T$ elements of $X$, problems with short aftereffect (no more than $T$) is still solvable.
		
		\item[Problems without optimal substructures.] Some problems (e.g., blackhole) have overlapping subproblems but no optimal substructures. Even though they are not DP problems, \tool still optimizes them by memoizing solutions to subproblems for data reuse. Instead of saving data as ``$M[i, c_i]=extreme\text{ }o_i(\cdot)$'', \tool saves ``$M[i, c_i, j]=true/false$'' to indicate ``\emph{given $i$ elements and constraint value $c_i$, whether or not $o_i$ is equal to j.''}. 
		All saved data will be enumerated when \tool searches for the best solution. 
		%No existing dynamic programming generation method is able to solve this type of problems. Our approach translates such a problem to a SAT problem: instead of finding the optimal value of the objective function, we figure out all possible values of the function, then choose the best one at the end of the program.
		
		\item[Problems with two or more candidate variables.] 
		Some problems (see Figure~\ref{fig:knapsack3}) have multiple candidate variables for DP optimization. 
		%For some problems, solving algorithms may be composed by multiple dynamic programming algorithms. 
		%Prior DP program generation approaches~\cite{Moor1994Categories,Sauthoff2011Bellman,Morihata2014Dynamic} cannot deal with such problems. In comparison, 
		\tool enumerates and assesses different optimization strategies by 
		concatenating arrays. With more detail, given multiple array variables B=\{$b_1$, $b_2$, \ldots, $b_k$\}, \tool enumerates subsets in $B$. For instance, if $k=3$, \tool first analyzes each array for any optimization opportunity. Next, \tool 
enumerates all array pairs for concatenation (e.g., $(b_1, b_2)$, $(b_2, b_3)$, and $(b_1, b_3)$), and analyzes whether any combined array can be used for optimization. Finally, \tool concatenates all arrays to obtain a larger one (e.g., $(b_1, b_2, b_3)$), and decides whether any optimization is applicable. Such analysis can be time-consuming when $k$ is large. In such scenarios, \tool ranks arrays based on their lengths and focuses the analysis on longer  arrays.
		%is able to solve problems containing multiple dynamic programming algorithms by dealing with not only a single array but also a set of multiple arrays.
	\end{description}


\section{Evaluation}

    \begin{table*}[htb]\centering
    \caption{Constraint solving time (in second) of different tools 
    %solving constraint models of dynamic programming problems
    }\label{timecost}
    \scriptsize
		\begin{tabular}{l|l|p{3cm}| r| r| r| r| r| r|r}\toprule
			\multirow{2}{*}{No.}&  \multirow{2}{*}{Problem}	&
			\multirow{2}{*}{Extra Handling} &\multirow{2}{*}{Size of Input}	&	\multirow{2}{*}{Gecode}	&   \multirow{2}{*}{\textsc{Chuffed}}    & 
			\multirow{2}{*}{\textsc{ChuffedC}}
			& \multirow{2}{*}{\textsc{ChuffedL}} &	\multicolumn{2}{c}{\tool}	\\	%Handwritten
			& & & & & & & &Total & Analysis\\
			\midrule
			1 & 0-1 knapsack &-	&$N=10^3, C=10^3$	&	975	& 1143  &  47&  1598  &	2	&	0.05	\\ 
			
			2 & Complete knapsack &	- & $N = 10^3, C = 10^3$ &	3985	&  4262    & 125	&$>$10000 &2 & 0.05	\\
			
			3 & Nested 0-1 knapsack & With multiple candidates &$N=100, C=100$ &357 & 496& 78 &672 & 1 & 0.03\\ 
			
			4 & Shortest path & With aftereffects (T=1) & $N=10^3$ & $>$10000 & $>$10000 & 784& $>$10000 & 97& 0.15\\
			5 & Longest inc. subseq. & With aftereffects (T=1)	& $N = 10^3$ &1274	& 1653  &	34 & 1751 &	5 & 0.03	\\

			6 & Longest common subseq. & With multiple candidates, with aftereffects (T=1) &  $N=10^3$ & $>$10000 & $>$10000  & 55& $>$10000 & 10& 0.08\\
           
           7 & Radiation therapy & With multiple candidates, with aftereffects (T=1)
&$M=8, N=10$& $>$10000& $>$10000& 40 &4 &4 & 0.18\\ 
			
			8 & Modulus 0-1 knapsack & Without P1 & $N=10^3, C=10^3$ & $>$10000 & $>$10000 & 58& $>$10000 & 53 & 0.05\\


			%Tricky 0-1 Knapsack &	$N = 10^3, C = 10^4$	&	7245.29	&   1102.13    &	12.92	&	1.22	\\
			
			%Longest Common Subseq.	& $N = 10^4$ &	$>$ 10000	&    681.03   &	9.24	&	2.33\\
			%Shortest path &With aftereffects ($T=1$)	& $N = 10^3$ &$>$10000	& & &783.62  &97.44&0.15	\\
			
			
			9 & Blackhole & Without P1, with multiple candidates &$N=52$ &168& 74& 46 &267 & 43 & 0.16\\
			%6 & Graph coloring & T=N & $N=50,C=10$ & $>$10000 &$>$10000 & 742.52 & 45.87& $>$10000& 0.03\\
			\bottomrule
		\end{tabular}
	
	\end{table*}

To evaluate the effectiveness of \tool, we created a data set of 9 representative COP tasks:  %(see \url{http://todo}):  
%Due to the space limit, here we only present and explain the evaluation results for six representative tasks: 
\begin{itemize}
    \item 0-1 knapsack.
    \item Complete knapsack: Similar to 0-1 knapsack but  each item can be selected multiple times. 
    \item Nested 0-1 knapsack.
    \item Shortest path: 
    Given a graph of $N$ nodes and the weighted edges between them, 
     find a path between $S$ and $T$ 
    to minimize the weight sum. 
    \item Longest increasing subsequence:
    Find the longest ascending subsequence in a given sequence of length $N$.  
    \item Longest common subsequence: Find the longest common subsequence between two given sequences of length $N$.
\item Radiation therapy~\cite{Baatar:2007}:
    Given an $M\times N$ integer matrix describing the radiation dose to deliver to each area, decompose the matrix into 
   a set of patterns for a radiation source to deliver. 
   Minimize the working time of the radiation source and the number of used patterns.
    \item Modulus 0-1 knapsack: Similar to 0-1 knapsack but the objective is to maximize the last digit of the value sum. 
    
   %Minimize the amount of time the source has to be switched on, as well as the number of patterns used (setup time of machine).
    \item Blackhole: The poker game is begun by moving Ace of Spades to the blackhole,
    and arranging the other 51 cards as 17 piles of 3 cards each. 
    In each turn, a card at the top of any pile can be moved to the blackhole if it is +1/-1 from the previous one. Find a way to move all cards.
    %\item Graph coloring: Color the N nodes of a graph with C colors such that no adjacent nodes have the same color.
\end{itemize}
The first seven problems are DP problems, and the last two problems are non-DP ones because they do not have the P1 property. The DP structures of those problems are distinct, and they are representative and cover most of the common DP structures from DP exercises.
We applied \tool, Gecode~\cite{schulte2006gecode}, and \textsc{Chuffed}~\cite{chu2012exploiting} to the data set. Specifically,
%Gecode is a widely used state-of-the-art constraint solver. 
\textsc{Chuffed} can be run as a na\"ive solver (denoted as \textsc{Chuffed}), a solver with automatic caching (\textsc{ChuffedC}), or a solver with Lazy Clause Generation~\cite{ohrimenko2009propagation}. Although both \textsc{ChuffedC} and \textsc{ChuffedL} reduce redundant search, \textsc{ChuffedC} reuses active and non-satisfied constraints while \textsc{ChuffedL} uses constraints involved in conflicts to generate new constraints (nogoods). 
%to exploit subproblem dominance (\textsc{ChuffedL}). 
We conducted the experiment 
on a personal computer (with an Intel Core i5-7300HQ 2.5GHz CPU, 8G of RAM). For generality, given the fixed parameter values of each problem, we randomly generated 10 instances, executed these instances for each tool, and reported the average runtime overhead between the 10 executions.

Table~\ref{timecost} shows the runtime overhead of different solvers. 
We set 10000 seconds for execution timeout. If a solver does not return any result within 10000 seconds, we terminated the execution. Because Gecode and \textsc{Chuffed} applied almost no optimization or only built-in negligible optimizations, both approaches ran overtime for four problems (No.~4 and 6-8). In contrast, \tool solved each of these problems within 100 seconds. 
For the remaining problems, \tool achieved on average 698x speedup over Gecode and 790x speedup over \textsc{Chuffed}. \textbf{\tool outperformed general constraint solvers with a speedup of several hundreds. }

We compared \tool with two optimized solvers: \textsc{ChuffedC} and \textsc{ChuffedL}. Unexpectedly, \textsc{ChuffedL} ran overtime for four problems (No.~2, 4, 6, and 8) while \tool solved all of them. 
For four out of the other problems, \textsc{ChuffedL} executed more slowly than both unoptimized solvers. This may be because a lot of nogoods are generated and propagated based on conflicts, but these nogoods could not help reduce the search space. 
For these five problems, \tool obtained 345x speedup over \textsc{ChuffedL} on average.
When solving all 9 problems, \tool achieved 23x speedup
over \textsc{ChuffedC}. %For the last two problems with P2 instead of P1, \tool worked similarly to \textsc{ChuffedC} but caching and reusing solutions to subproblems.  
\textbf{Our approach outperformed caching and LCG when solving DP problems and some non-DP problems with the P2 property.}

Actually, the execution time of \tool is spent for two tasks: (1) model analysis (including transformation), and (2) constraint solving. To compare the benefit and cost of model analysis, we measured the analysis time cost. 
As shown by the last column of Table~\ref{timecost}, \tool spent 0.03-0.18 second in analyzing each model and generating new models when necessary. The analysis time cost is negligible compared with the constraint solving time \tool saved over existing techniques. \textbf{Our approach significantly accelerated constraint solving without introducing substantial analysis overhead.}    
	

\section{Related Work}
This section describes related work on automatic optimization of constraint solving (Section~\ref{sec:opt_related}), constraint solving with DP (Section~\ref{sec:dp_related}), and tabling (Section~\ref{sec:table_related}). 
    
\subsection{Optimizers of Constraint Solving}\label{sec:opt_related}
When constraint solvers are used to search solutions for COP tasks, 
a number of methods were proposed to avoid or reduce redundant search in the process~\cite{Smith2005,Gent2006,kitching2007symmetric,ohrimenko2009propagation,chu2012exploiting}. Specifically, 
caching memoizes search states to prevent the same state from being recomputed~\cite{Smith2005}. 
Symmetry breaking and subproblem dominance identify the equivalence or dominance relationship between states to avoid useless state  exploration~\cite{Gent2006,chu2012exploiting}. Symmetric component caching decomposes a COP into disjoint subproblems via variable assignments, such that each subproblem is solved independently to compose the optimal solution~\cite{kitching2007symmetric}. 
Lazy Clause Generation (LCG) identifies constraints related to conflicts, and generates new constraints (nogoods) to reduce search~\cite{ohrimenko2009propagation}.

Compared with prior work, \tool reduces unnecessary search by (1) identifying optimal substructures and overlapping subproblems in given problem descriptions, and (2) formatting the descriptions as DP-oriented models when possible. Our evaluation shows that \tool worked much better than the state-of-the-art optimizers.

\subsection{Constraint Solving with DP}
\label{sec:dp_related}
Researchers created various methods to solve constraints via DP~\cite{Moor1994Categories,Sauthoff2011Bellman,Morihata2014Dynamic,prestwich2018towards}.
For instance, DPE is a method to model DP in Constraint Programming (CP)~\cite{prestwich2018towards}. With this method, a modeler can specify a DP problem by defining a CP model via KOLMOGOROV---a constraint specification language. 
GAP is another domain-specific language for DP problem specification~\cite{Sauthoff2011Bellman}. However, 
none of these approaches can automatically detect any DP problem structure in general constraint models.

Moor proved the conditions in which we can check monotonicity to verify the optimal substructures of given problems~\cite{Moor1994Categories}. The research inspires our approach of DP problem recognition.
Morihata et al.~built a Haskell library for users to describe a problem in a naive enumerate-and-choose style, and provided an approach that automatically derives efficient algorithms to solve the problem~\cite{Morihata2014Dynamic}. However, the library does not support users to specify any problem with multiple constraints, neither can users specify the three classes of problems listed in Section \ref{sec:impl}.


%It cannot detect any DP problem structure in general constraint models.
%In comparison, DPSolver detects any DP structure in a constraint model, and reformulates the model as needed.


\subsection{Tabling}
\label{sec:table_related}

Approaches were built to save intermediate computation results in certain data structures, so as to reduce or eliminate repetitive computation~\cite{zhou2015constraint,dekker2017auto,de2019compiling}. For example, 
auto-tabling provides MiniZinc annotations for users to define and insert table constraints and save calculation results to tables~\cite{dekker2017auto}.
Similarly, the Picat tabling requires users to specify the variables for tabling~\cite{zhou2015constraint}. Different from these tools, DPSolver automatically detects memoization opportunities and creates tables without any user input.

\citeauthor{de2019compiling} detected subproblem equivalence by hashing results~\cite{chu2012exploiting}, and then used MDDs and formulas in d-DNNFs instead of tables to compute and store solutions~\cite{de2019compiling}. To build an MDD or d-DNNF, a modeler has to specify the constraints, and the domain as well as ordering of related variables. DPSolver is different in two ways. First, it unfolds and analyzes predicates to detect overlapping subproblems and/or optimal substructures. Second, it does not require extra user input.
\section{Discussion}

Theoretically, \tool is applicable when a model $\mathcal{M}$ meets two criteria:

\begin{enumerate}
\item[(i)] $\mathcal{M}$ describes a discrete optimization problem that has overlapping subproblems; and
\item[(ii)] $\mathcal{M}$ has at least an array or set of same-typed variables.
\end{enumerate}
Essentially, (i) ensures the opportunity to trade space for time. Namely, if there is redundant computation between subproblems, we can save computed results to remove duplicated calculation.
Additionally, (ii) ensures the applicability of the table-based search. With a fixed linear ordering between variables, we can index and retrieve solutions to subproblems.


In practice, \tool’s capability is also limited by another two conditions:

\begin{enumerate}
\item[(iii)] All constraints in $\mathcal{M}$ are defined with the built-in operations, global constraints, and non-recursive functions provided by MiniZinc. This is imposed by our current tool implementation.
\item[(iv)] The available memory space allocated by users should be sufficient to hold a table. Suppose that
\begin{itemize}
\item[a)] there are $N$ elements in a candidate array;
\item[b)] the size of each element's value range is $M$;
\item[c)] there are $L$ ($L \geq 1$) accumulative functions related to the array;
\item[d)] for the $i^{th}$ accumulative function, the size of the constraint value's range is $R_i$; and
\item[e)] the aftereffect value is $T$.
\end{itemize}
Then the space complexity is $O(N M^T \prod\limits_{i=1}^L R_i)$. With this formula, \tool decides whether the complexity value is larger than the allocated space; if not, \tool optimizes the model.
\end{enumerate}
\tool is not limited to solving ``pure'' DP problems; 
it can handle some non-DP problems as long as the memoization of intermediate results can reduce duplicated calculation. 
For instance, the graph coloring problem is about coloring the $N$ nodes in a graph with $C$ different colors such that no two adjacent nodes have the same color. 
This non-DP problem can be described as a DP-oriented model with $T=N$. \tool optimizes the solving process when $N$ is small (e.g., $N=10$), but does not do so when $N$ is large (e.g., 50) because the space cost is too high.

Furthermore, 
we manually checked the 130 models in MiniZinc benchmarks~\cite{min}, to estimate the applicability of our approach. Based on our observation,   
\tool can fully optimize 9 models, and partially optimize 44 models by working for smaller sizes of inputs. The other problems have complicated constraints, and can be potentially handled by \tool if we simplify the constraints.

%Graph Coloring represents another kind of non-DP problems that can be handled by \tool. By definition, ``Graph Coloring'' is about coloring the $N$ nodes in a graph with $C$ different colors such that no two adjacent nodes have the same color. 

%By modeling it as a DP problem with aftereffects $T=N$, \tool can optimize the constraint solving when $N$ is small, but gives up the optimization when $N$ is so large that lots of memory are required to store the value assignments of the previous $(N-1)$ elements

%For some DP problems with aftereffects, \tool may or may not optimize constraint solving depending on the input data size. For instance, ``Graph Coloring'' is about coloring the $N$ nodes in a graph with $C$ different colors such that no two adjacent nodes have the same color.

%When we model this problem as a DP problem, $T=N$, meaning that for each subproblem containing $n$ nodes, we should record the optimal color assignment 

%	However, our approach fails to solve dynamic programming on tree, such as Matrix Chain Multiplication. That is because we can only process linear arrays. And it is not easy to describe tree structures by constraints.

%	In this paper, we focus on generating dynamic programming algorithms to solve subproblems in constraint models more efficiently. Dynamic programming algorithms trade time with space, so the use of our approach should take the trade-off between time and space into consideration.

%    Our approach performs static analysis and generates the the dynamic programming algorithms statically. Therefore, our approach is naturally incremental: we only need to generate the algorithms once and then we can deal with many different inputs without regenerating the program. However, which algorithm is selected to use is influenced by the values of the input arguments.
\section{Conclusion and Future Work}

\tool opportunistically optimizes constraint solving if a problem (1) uses any array variable and (2) can be reformulated as a DP-oriented problem. We focused on dynamic programming because although it is a popularly used problem-solving paradigm for large problems, it has not been exploited to efficiently solve constraints. 

Our research has made three contributions. First, \tool analyzes a given problem modeled with MiniZinc to automatically check for two DP-related properties. Second, \tool converts problem descriptions to DP-oriented models in novel ways such that DP problems and some non-DP problems can be be efficiently resolved by generic constraint solvers. 
Third, we applied \tool and related techniques to nine representative COP problems (including DP and non-DP problems). Impressively, our evaluation demonstrates that \tool outperforms current tools with dozens of or even hundreds of speedups.  

Our investigation demonstrates the effectiveness of accelerating constraint solving via dynamic programming. \tool currently handles problems containing array variables. In the future, we will improve \tool to also handle problems having variables of other data structures, such as trees. 
Given a problem description written in natural languages, we also plan to automatically create a DP-oriented model by extracting arguments, variables, constraints, and objective functions directly from the description. In this way, users will learn about which problem is efficiently solvable and what is the corresponding MiniZinc model. 
%problem description written in natural languages. Several works on finding approximate optimal solutions by reinforcement learning are undergoing, which are based on the subproblem construction method described in this paper.

%	We have introduced our approach that generates dynamic programming algorithms to deal with arrays in constraint solving. In our approach, problems that is written in a general constraint modeling language will be processed and analyzed automatically. We design subproblems and generate recursive formulas and verify the optimization structure, then generate the program and estimate the space and time cost. The evaluation results show that our approach can be widely applied on pure dynamic programming problems, and is more efficient than existing constraint solvers.

%	We are planning to make a further step on optimizing our approach to solve more types of dynamic programming programs, including dynamic programming on tree, or dynamic programming optimized by data structures. 


%%
%% Bibliography
%%

%% Please use bibtex, 


\bibliographystyle{named}
\bibliography{linshu}

\end{document}

