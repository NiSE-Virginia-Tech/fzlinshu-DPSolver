\section{Discussion}

Theoretically, \tool is applicable when a model $\mathcal{M}$ meets two criteria:

\begin{enumerate}
\item[(i)] $\mathcal{M}$ describes a discrete optimization problem that has overlapping subproblems; and
\item[(ii)] $\mathcal{M}$ has at least an array or set of same-typed variables.
\end{enumerate}
Essentially, (i) ensures the opportunity to trade space for time. Namely, if there is redundant computation between subproblems, we can save computed results to remove duplicated calculation.
Additionally, (ii) ensures the applicability of the table-based search. With a fixed linear ordering between variables, we can index and retrieve solutions to subproblems.


In practice, \tool’s capability is also limited by another two conditions:

\begin{enumerate}
\item[(iii)] All constraints in $\mathcal{M}$ are defined with the built-in operations, global constraints, and non-recursive functions provided by MiniZinc. This is imposed by our current tool implementation.
\item[(iv)] The available memory space allocated by users should be sufficient to hold a table. Suppose that
\begin{itemize}
\item[a)] there are $N$ elements in a candidate array;
\item[b)] the size of each element's value range is $M$;
\item[c)] there are $L$ ($L \geq 1$) accumulative functions related to the array;
\item[d)] for the $i^{th}$ accumulative function, the size of the constraint value's range is $R_i$; and
\item[e)] the aftereffect value is $T$.
\end{itemize}
Then the space complexity is $O(N M^T \prod\limits_{i=1}^L R_i)$. With this formula, \tool decides whether the complexity value is larger than the allocated space; if not, \tool optimizes the model.
\end{enumerate}
\tool is not limited to solving ``pure'' DP problems; 
it can handle some non-DP problems as long as the memoization of intermediate results can reduce duplicated calculation. 
For instance, the graph coloring problem is about coloring the $N$ nodes in a graph with $C$ different colors such that no two adjacent nodes have the same color. 
This non-DP problem can be described as a DP-oriented model with $T=N$. \tool optimizes the solving process when $N$ is small (e.g., $N=10$), but does not do so when $N$ is large (e.g., 50) because the space cost is too high.

Furthermore, 
we manually checked the 130 models in MiniZinc benchmarks~\cite{min}, to estimate the applicability of our approach. Based on our observation,   
\tool can fully optimize 9 models, and partially optimize 44 models by working for smaller sizes of inputs. The other problems have complicated constraints, and can be potentially handled by \tool if we simplify the constraints.

%Graph Coloring represents another kind of non-DP problems that can be handled by \tool. By definition, ``Graph Coloring'' is about coloring the $N$ nodes in a graph with $C$ different colors such that no two adjacent nodes have the same color. 

%By modeling it as a DP problem with aftereffects $T=N$, \tool can optimize the constraint solving when $N$ is small, but gives up the optimization when $N$ is so large that lots of memory are required to store the value assignments of the previous $(N-1)$ elements

%For some DP problems with aftereffects, \tool may or may not optimize constraint solving depending on the input data size. For instance, ``Graph Coloring'' is about coloring the $N$ nodes in a graph with $C$ different colors such that no two adjacent nodes have the same color.

%When we model this problem as a DP problem, $T=N$, meaning that for each subproblem containing $n$ nodes, we should record the optimal color assignment 

%	However, our approach fails to solve dynamic programming on tree, such as Matrix Chain Multiplication. That is because we can only process linear arrays. And it is not easy to describe tree structures by constraints.

%	In this paper, we focus on generating dynamic programming algorithms to solve subproblems in constraint models more efficiently. Dynamic programming algorithms trade time with space, so the use of our approach should take the trade-off between time and space into consideration.

%    Our approach performs static analysis and generates the the dynamic programming algorithms statically. Therefore, our approach is naturally incremental: we only need to generate the algorithms once and then we can deal with many different inputs without regenerating the program. However, which algorithm is selected to use is influenced by the values of the input arguments.